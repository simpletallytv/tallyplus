\documentclass[12pt]{article}
%\title{LTC Reader Owner's Manual}\label{ltc-reader-owners-manual}
\title{\textbf{TallyPlus руководство пользователя}}
\date{} % Устанавливаем пустую дату
\author{}


\usepackage[a4paper, left=1.5cm, right=1.5cm, top=1cm, bottom=1.5cm]{geometry}
%\usepackage{hyperref} % Загружаем пакет для гиперссылок
\usepackage{graphicx}
\usepackage{enumitem} % Подключаем пакет для работы со списками
\usepackage{longtable} % Подключаем пакет для длинных таблиц
\usepackage{booktabs}  % Подключаем пакет для красивых таблиц
\usepackage{tabularx}  % Для управления шириной столбцов
\usepackage{fancyhdr} % Для пользовательских колонтитулов
\usepackage{float}
\usepackage{tcolorbox} % для callout
\usepackage{mdframed} % для callout
\usepackage{fontspec}
\usepackage{colortbl}
\usepackage{array} % выравнивание в таблице по вертикали

\usepackage[hidelinks]{hyperref} % Убрать цветные рамки вокруг ссылок

% Устанавливаем шрифт Open Sans
\setmainfont{Open Sans}


% Переопределяем команды для заголовка
%\renewcommand{\title}[1]{\begin{center}\textbf{\huge #1}\end{center}}
%\renewcommand{\author}[1]{\begin{center}\textbf{#1}\end{center}}
%\renewcommand{\date}[1]{\begin{center}\textbf{#1}\end{center}}

\begin{document}
	
	\begin{titlepage} % Титульный лист
		\centering
		\maketitle
		\thispagestyle{empty} % Убирает номер страницы на первой странице
		\vspace{1cm} % Отступ перед изображением
		\begin{figure}[h]
			\centering
			\includegraphics[width=18cm, keepaspectratio]{imgs/title.png}
		\end{figure}
		\vfill % Отодвигает текст вниз страницы
		% Текст внизу страницы
		\Large
		Декабрь 2025 \\
		SimpleTally \\
		\href{http://www.simpletally.tv}{www.simpletally.tv}
	\end{titlepage}
	\pagenumbering{arabic} % Устанавливает арабскую нумерацию
	\setcounter{page}{2}   % Устанавливает номер страницы на 1 для следующей страницы	
	\tableofcontents % Вставка оглавления
	\newpage % Переход на новую страницу после оглавления
	
	\section{Введение}\label{about}
	Устройство предназначено для работы совместно с системными и портативными камерами в аппаратно-студийных комплексах.
	\noindent Назначение: 
	\begin{itemize}
		\item Идентификация номера камеры
		\item On-Air сигнализация
	\end{itemize}
	\noindent Управление On-Air сигнализацией осуществляется одим из способов:
	\begin{itemize}
		\item LDR сенсор (датчиком освещенности)
		\item Сигнал GPI (General Purpose Interface)	
	\end{itemize}
	
	\section{Дизайн}\label{design}
	Устройство имеет:
	\begin{itemize}
		\item Микропроцессорное управление
		\item Настройка номера индикации
		\item Настройка яркости индикации
		\item Настройка чувствительности датчика освещенности
		\item Энергонезависимая память для хранения настроек
	\end{itemize}
	%\vspace{1cm} % Отступ перед изображением
	\begin{figure}[H]
		\centering
		\includegraphics[height=4cm]{imgs/iso_left_1.png}
		\includegraphics[height=4cm]{imgs/iso_right_1.png}
	\end{figure}

	\section{Комплектация}
	\begin{itemize}
		\item Индикатор TallyPlus / TallyPlus Twin— 1шт
		\item Блок питания с разъемом GX12 — 1шт
		\item LDR сенсор с разъемом mini-XRL — 1шт
		\item Кабель питания и управления — опция
		\item Адаптер крепления — опция
	\end{itemize}
	
	\section{Спецификация}
	\begin{center}
		\begin{tabular}{|m{0.02\textwidth}|m{0.3\textwidth}|m{0.65\textwidth}|} \hline
			\bfseries № & \bfseries Наименование & \bfseries Значение \\ \hline
			1 & Количество семисегментных индикаторов, шт & 1 (TallyPlus),
													      \newline 2 (TallyPlus Twin) \\ \hline
			2 & Диапазон индикации & 0-9 (TallyPlus),
			                                       \newline 00-99 (TallyPlus Twin) \\ \hline
			3 & Цвет свечения & зеленый (обычное состояние),
							    \newline красный (активировано сигналом управления) \\ \hline
			4 & Количество уровней яркости, шт & 6 \\ \hline
			5 & Сигнал управления & LDR (фоторезистор) или GPI (+3 - +6 В, потенциометр должен быть в крайнем левом положении) \\ \hline
			6 & Крепление & сверху и снизу резъба 1/4″-20 UNC (подходят стандартные аксессуары для фототехники) \\ \hline
			7 & Электропитание, В & 9 – 18 \\ \hline
			8 & Потребляемая мощность, Вт & 0.5 \\ \hline
			9 & Способ подачи питания & от камеры или от внешнего блока питания \\ \hline
			10 & Размер без разъемов и органов управления (ШхВхГ), мм & 70x93x45 (TallyPlus),
			                          \newline 118x93x45 (TallyPlus Twin) \\ \hline
			11 & Размер упаковки (ШхВхГ), мм & 118x103x102 (TallyPlus),
										\newline 168x103x102 (TallyPlus Twin) \\ \hline
			12 & Вес, гр & 254 (TallyPlus),
			               \newline 354 (TallyPlus Twin) \\ \hline
			13 & Вес с упаковкой, гр & 600 (TallyPlus),
			                           \newline 750 (TallyPlus Twin) \\ \hline
			
		\end{tabular}
	\end{center}	
	
	\section{Варианты подключения}
	\begin{center}
		\begin{tabular}{|m{0.02\textwidth}|m{0.4\textwidth}|m{0.25\textwidth}|m{0.25\textwidth}|} \hline
			\bfseries № & \bfseries Варианты подключения & \bfseries Разъем GX12 & \bfseries Разъем miniXLR \\ \hline
			1 & Питание и управление от камеры & Питание и GPI & не используется \\ \hline 
			2 & Питание от камеры, управление от датчика LDR (фото резистор) & Питание & LDR \\ \hline
			3 & Питание от внешнего блока питания, управление от камеры или внешнего & Питание & GPI \\ \hline
		\end{tabular}
	\end{center}
	% \newpage % Переход на новую страницу
		
	\section{Порядок работы}
		\begin{enumerate}
			\item Установить и зафиксировать индикатор на камере или телесуфлере. Для крепления использовать фото аксессуары с резьбовым креплением 1/4” дюйма.
			\item Произвести подключение кабелей в зависимости от выбранного варианта подключения
			\item С помощью манипуляторов размещенных на правой стороне индикатора настроить номер и яркость свечения индикатора. Манипуляторы - 3-x позиционные с фиксацией в среднем положении.
			\item При режиме работы с датчиком LDR (фоторезистор), с помощью регулятора чувствительности, размещенного на боковой стороне настроить порог переключения цвета индикации. При работе от сигнала GPI установить регулятор чувствительности в крайнее минимальное положение (повернуть против часовой стрелки до щелчка)
			\item Проверить корректность работы индикатора переключая камеру в режимы On-Air / Off-Air.
		\end{enumerate}
	\section{Подключение кабелей}
		\subsection{GX12}
			\begin{center}
				\begin{tabular}{|c|c|} \hline
					\bfseries Контакт & \bfseries Сигнал \\ \hline
					1 & GND \\ \hline
					2 & GPI \\ \hline
					3 & GND \\ \hline
					4 & POW  \\ \hline
				\end{tabular}		
			\end{center}
		\subsection{miniXLR}
			\begin{center}
				\begin{tabular}{|c|c|} \hline
					\bfseries Контакт & \bfseries Сигнал \\ \hline
					1 & LDR  \\ \hline
					2 & не используется \\ \hline
					3 & GPI \\ \hline
					4 & GND  \\ \hline
				\end{tabular}		
			\end{center}
	
	\section{Информация для заказа}
		\begin{center}
			\begin{tabular}{|m{0.05\textwidth}|m{0.25\textwidth}|m{0.65\textwidth}|} 
				\hline
				\bfseries № & \bfseries Артикул & \bfseries Описание \\ \hline
				1 & TallyPlus & Индикатор односегментный \\ \hline
				2 & TallyPlus Twin & Индикатор двухсегментный \\ \hline
				3 & TallyPlus H4p100 & Кабель питания и управления GX12 — Hirose 4-pin male для подключения к разъему Script light на Grass Valley LDX 98, длина 1 метр \\ \hline
				4 & TallyPlus H20p100 & Кабель питания и управления GX12 — Hirose 20-pin male для подключения к разъему Auxiliary на Grass Valley LDX 98, длина 1 метр \\ \hline
				5 & TallyPlus F1 & Адаптер крепления для телесуфлера Fortinge ERA IP \\ \hline
				6 & TallyPlus F2 & Адаптер крепления универсальный \\ \hline
			\end{tabular}
		\end{center}
	
	\section{Контакты}\label{how-to-contact-us}
		Сайт \textbf{\href{http://www.simpletally.tv}{www.simpletally.tv}}
	
\end{document}
