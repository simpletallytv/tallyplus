\documentclass{article}
%\title{LTC Reader Owner's Manual}\label{ltc-reader-owners-manual}
\title{\textbf{TallyPlus руководство пользователя}}
\date{} % Устанавливаем пустую дату
\author{}


\usepackage[a4paper, left=2cm, right=2cm, top=2cm, bottom=2cm]{geometry}
%\usepackage{hyperref} % Загружаем пакет для гиперссылок
\usepackage{graphicx}
\usepackage{enumitem} % Подключаем пакет для работы со списками
\usepackage{longtable} % Подключаем пакет для длинных таблиц
\usepackage{booktabs}  % Подключаем пакет для красивых таблиц
\usepackage{tabularx}  % Для управления шириной столбцов
\usepackage{fancyhdr} % Для пользовательских колонтитулов
\usepackage{float}
\usepackage{tcolorbox} % для callout
\usepackage{mdframed} % для callout
\usepackage{fontspec}
\usepackage{colortbl}

\usepackage[hidelinks]{hyperref} % Убрать цветные рамки вокруг ссылок

% Устанавливаем шрифт Open Sans
\setmainfont{Open Sans}


% Переопределяем команды для заголовка
%\renewcommand{\title}[1]{\begin{center}\textbf{\huge #1}\end{center}}
%\renewcommand{\author}[1]{\begin{center}\textbf{#1}\end{center}}
%\renewcommand{\date}[1]{\begin{center}\textbf{#1}\end{center}}

\begin{document}
	
	\begin{titlepage} % Титульный лист
		\centering
		\maketitle
		\thispagestyle{empty} % Убирает номер страницы на первой странице
		\vspace{1cm} % Отступ перед изображением
		\begin{figure}[h]
			\centering
			\includegraphics[width=\textwidth, keepaspectratio]{imgs/title.png}
		\end{figure}
		\vfill % Отодвигает текст вниз страницы
		% Текст внизу страницы
		\Large
		Июль 2025 \\
		SimpleTally \\
		\href{http://www.simpletally.tv}{www.simpletally.tv}
	\end{titlepage}
	\pagenumbering{arabic} % Устанавливает арабскую нумерацию
	\setcounter{page}{2}   % Устанавливает номер страницы на 1 для следующей страницы	
	\tableofcontents % Вставка оглавления
	\newpage % Переход на новую страницу после оглавления
	
	\section{Введение}\label{about}
	Устройство предназначено для работы совместно с системными и портативными камерами в аппаратно-студийных комплексах.
	\noindent Назначение: 
	\begin{itemize}
		\item Идентификация номера камеры
		\item On-Air сигнализация
	\end{itemize}
	\noindent Управление On-Air сигнализацией осуществляется одим из способов:
	\begin{itemize}
		\item LDR сенсор (датчиком освещенности)
		\item сигнал GPI (General Purpose Interface)	
	\end{itemize}
	
	\section{Дизайн}\label{design}
	Лампа имеет:
	\begin{itemize}
		\item микропроцессорное управление
		\item настройка номера индикации
		\item настройка яркости индикации
		\item настройка чувствительности датчика освещенности
		\item энергонезависимая память для хранения настроек
	\end{itemize}

	\section{Комплектация}
	\begin{itemize}
		\item Индикатор Tally Plus — 1шт
		\item Блок питания с разъемом GX12 — 1шт
		\item LDR сенсор с разъемом mini-XRL — 1шт
		\item Кабель питания и управления — опция
		\item Адаптер крепления — опция
	\end{itemize}
	
	\section{Спецификация}
	\begin{center}
		\begin{tabular}{|p{0.1\textwidth}|p{0.4\textwidth}|p{0.5\textwidth}|} \hline
			№ & Наименование & Значение \\ \hline
			1 & Количество семисегментных индикаторов, шт & 1 (TallyPlus), 2 (TallyPlus Dual Digit) \\ \hline
			2 & Диапазон индикации номера камеры & 0-9 (TallyPlus), 00-99 (TallyPlus Dual Digit) \\ \hline
			3 & Цвет свечения & зеленый (обычное состояние), красный (активировано сигналом управления) \\ \hline
			4 & Количество уровней яркости, шт & 6 \\ \hline
			5 & Сигнал управления & GPI или LDR (фоторезистор) \\ \hline
			6 & Крепление & сверху и снизу резъба 1/4″-20 UNC (подходят стандартные аксессуары для фототехники) \\ \hline
			7 & Электропитание, В & 9 – 18V \\ \hline
			8 & Потребляемая мощность, Вт & 0.5 \\ \hline
			9 & Способ подачи питания & от камеры или от внешнего блока питания \\ \hline
			10 & Размер (ШхВхГ), мм & 69x92x44 (TallyPlus), 130x92x44 (TallyPlus Dual Digit) \\ \hline
			11 & Вес, гр & 100 (TallyPlus), 150 (TallyPlus Dual Digit) \\ \hline
		\end{tabular}
	\end{center}	
	
	\section{Варианты подключения}
	\begin{center}
		\begin{tabular}{|p{0.1\textwidth}|p{0.4\textwidth}|p{0.2\textwidth}|p{0.2\textwidth}|} \hline
			№ & Варианты подключения & Разъем GX12 & Разъем miniXLR \\ \hline
			1 & Питание и управление от камеры & Питание и GPI & не используется \\ \hline 
			2 & Питание от камеры, управление от датчика LDR (фото резистор) & Питание & LDR \\ \hline
			3 & Питание от внешнего блока питания, управление от камеры или внешнего & Питание & GPI \\ \hline
		\end{tabular}
	\end{center}
		
	\section{Порядок работы}
		\begin{enumerate}
			\item Установить и зафиксировать индикатор на камере или телесуфлере. Для крепления использовать фото аксессуары с резьбовым креплением 1/4” дюйма.
			\item Произвести подключение кабелей в зависимости от выбранного варианта подключения
			\item С помощью манипуляторов размещенные на правой боковой стороне индикатора настроить номер и яркость свечения индикатора. Манипуляторы - 3-x позиционные с фиксацией в среднем положении.
			\item Проверить корректность работы индикатора переключая камеру в режимы On-Air / Off-Air.
		\end{enumerate}
	\section{Подключение кабелей}
		\subsection{GX12}
			\begin{center}
				\begin{tabular}{|c|c|} \hline
					Контакт & Сигнал \\ \hline
					1 & GND \\ \hline
					2 & GPI \\ \hline
					3 & GND \\ \hline
					4 & POW  \\ \hline
				\end{tabular}		
			\end{center}
		\subsection{miniXLR}
			\begin{center}
				\begin{tabular}{|c|c|} \hline
					Контакт & Сигнал \\ \hline
					1 & LDR  \\ \hline
					2 & GND \\ \hline
					3 & GPI \\ \hline
					4 & не используется  \\ \hline
				\end{tabular}		
			\end{center}
	
	\section{Контакты}\label{how-to-contact-us}
	Сайт \textbf{\href{http://www.simpletally.tv}{www.simpletally.tv}}
	
\end{document}
